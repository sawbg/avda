\hypertarget{README_8md_source}{\section{R\+E\+A\+D\+M\+E.\+md}
}

\begin{DoxyCode}
00001 # The Automatic Vasospasm Detection Application
00002 
00003 ## Introduction
00004 The Automatic Vasospasm Detection Application (or Algorithm, depending on the
00005 usage), AVDA, is an application to objectively detect the presence of vasospasms
00006 based on comparisons of parameters extracted from transcranial doppler audio.
00007 
00008 ## Setup
00009 AVDA is intended to be compiled on machines running Linux, though it could
00010 likely be adapter for other environments. It must be downloaded from GitHub.com
00011 and compiled locally. To do this, navigate to the directory in which AVDA should
00012 be placed, then execute the following commands
00013 
00014    git clone https://github.com/sawbg/avda
00015    cd avda
00016    make
00017 
00018 Sucessfully cloning, compilation, and execution of AVDA requires up-to-date
00019 versions of the following executables:
00020 
00021 * git
00022 * make
00023 * gcc (4.9)
00024 * arecord
00025 
00026 The recording device name used by arecord in AVDA will most likely need to be
00027 changed. In addition, the hard-coded patient CSV file path will need to be
00028 either created or changed before compilation.
00029 
00030 ## Demonstration
00031 A demonstration of AVDA can be seen on YouTube 
00032 [here](https://www.youtube.com/watch?v=HnQShhHQ\_M0).
00033 
00034 ## FAQ
00035 
00036 * **Why was this project developed?** This project was developed as a course 
00037 project by two gradute students at the University of Alabama at Birmingham
00038 School of Engineering, Nicholas Nolan and Andrew Wisner.
00039 
00040 * **Is AVDA an active project?** Though it is not planned to develop AVDA
00041 further in the near future, it is hoped that the algorithm discovered and
00042 implemented can be used and built upon by researchers to fully automate the
00043 detection of vasospasms.
00044 
00045 * **AVDA is returning unusually low or high parameters. Why might this be?** In
00046   development, this occurred when the mic-in volume was set too high. It is
00047 likely in this senario that clipping is happening or that the signal (or a
00048 strong enough signal) has no been received.
00049 
00050 * **How will AVDA be affected by the machine uprising?** The University
00051   supercomputer, Cheaha, has assured us that AVDA will not be needed after the
00052 uprising occures.
00053 
00054 * **What about more specific questions?** Questions relating to AVDA not
00055 covered in this FAQ may be sent to the AVDA team via awisner94@gmail.com.
\end{DoxyCode}
